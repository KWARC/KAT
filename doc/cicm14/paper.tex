\documentclass{llncs}
\usepackage[show]{ed}
\usepackage{calbf}
\usepackage{amstext,amssymb}
\usepackage{xspace}

\usepackage{mdframed}
\newenvironment{boxedquote}{\begin{mdframed}[leftmargin=1cm,rightmargin=1cm]}{\end{mdframed}}

\usepackage{wrapfig,paralist}
\usepackage[hyperref,style=alphabetic]{biblatex}
\addbibresource{kwarcpubs.bib}
\addbibresource{extpubs.bib}
\addbibresource{kwarccrossrefs.bib}
\addbibresource{extcrossrefs.bib}

\pagestyle{plain}
\usepackage{tikz}
\usetikzlibrary{shapes.geometric,docicon}

\def\defemph#1{\textbf{#1}}
\def\defeq{:=}
\def\omdoc{\textsf{OMDoc}\xspace}
\def\mmt{\textsf{MMT}\xspace}
\def\mathhub{\textsf{MathHub.info}\xspace}
\def\sys{\textsf{KAT}\xspace}
\def\KAT{\textsf{KAT}\xspace}
\usepackage[linkcolor=black,citecolor=black,urlcolor=black,colorlinks=true,breaklinks=true,
plainpages=false,pdfpagelabels]{hyperref}

\title{System Description: KAT an Annotation Tool for STEM Documents}
\author{Mircea Alex Dumitru, Deyan Ginev, Michael  Kohlhase, Vlad Merticariu, Stefan
  Mirea, and Tom Wiesing}
\institute{
  Computer Science\\ Jacobs University Bremen\\
  \url{http://kwarc.info}
}

\begin{document}
\maketitle
\begin{abstract}
  Current natural language understanding systems do not work particularly well on
  mathematical and technical documents as they cannot deal with formulae, diagrams, and
  the special, technical vocabularies and discourse conventions of such documents. To
  retrain existing tools and evaluate new ones specifically developed for STEM documents,
  we need to establish manually annotated document corpora. Unfortunately, even the
  annotation tools used in computational linguistics do not work well with mathematical
  documents, as they assume plain texts.

  This report presents the \sys system, browser-based annotation tool for
  linguistic/semantic annotations in structured (XHTML5) documents. As it is parametric in
  the annotation ontology and represents annotations as RDF, it can easily be integrated
  into RDF-based corpus management systems; we present an integration into the CorTeX
  system.
\end{abstract}

\section{Introduction}\label{sec:intro}


\section{System Architecture and Realization}\label{sec:arch}
\begin{figure}[ht]\centering
\def\localxscale{.8}\def\localyscale{1.1}
\documentclass{standalone}
\usepackage{tikz}
\usetikzlibrary{shapes.geometric,docicon}
\begin{document}
\providecommand\localyscale{1.5}
\providecommand\localxscale{1}
\begin{tikzpicture}[xscale=\localxscale,yscale=\localyscale]
  \tikzstyle{system} = [rectangle, draw, fill=blue!20, text centered,
                                    rounded corners, minimum height=1cm,shade, 
                                    top color=white, bottom color=blue!20]
  \tikzstyle{doc}=[draw,align=center,color=black,shape=document]
  \tikzstyle{database}=[cylinder,shape border rotate=90,aspect=0.25,draw, 
     cylinder uses custom fill,cylinder body fill=yellow!30,cylinder end fill=yellow!30]
  \pgfdeclareimage[width=1cm]{user}{user};
  \node (user) at (-1.2,2) {\pgfuseimage{user}};
  \node[doc] (kb) at (6,1.5) {\begin{tabular}{c}KAT\\Binding\end{tabular}};
  \node[doc] (onto) at (6,0) {\begin{tabular}{c}Anno-\\tation\\Onto-\\logy\end{tabular}};
  \node[database] (docstore) at (-.5,0) {\begin{tabular}{c}Document
      \\Store\\\hline tHTML5 \end{tabular}};
  \node[database] (semblack) at (3,0) {\begin{tabular}{c}Semantic
      \\Blackboard\\\hline RDF\end{tabular}};
  \node[system] (kat) at (2,2) {\begin{tabular}{c}KAT\\Annotator \end{tabular}};
  \draw[->,thick] (docstore) -- node[left]{import} (kat);
  \draw[->,thick] (semblack) to[out=100,in=-50] node[right]{import} (kat);
  \draw[->,thick] (kat) to[out=-80,in=120] node[left,near end]{export} (semblack);
  \draw[->,dotted,thick] (semblack) -- node[above]{ref} (docstore);
  \draw[<->,dashed] (user) -- node[above] {interact} (kat);
  \draw[->,thick] (kb) -- node[above]{read} (kat);
  \draw[->,dotted,thick] (kb) -- node[left]{ref} (onto);
  \draw[->,dotted,thick] (semblack) -- node[above]{ref} (onto);
\end{tikzpicture}
\end{document}
%%% Local Variables: 
%%% mode: latex
%%% TeX-master: t
%%% End: 

\caption{The \KAT System Architecture}\label{fig:kat-arch}
\end{figure}

\section{Conclusion}\label{sec:concl}

\ednote{MK@MK: say something} 
\ednote{Acknowledgements}


\printbibliography
\end{document}

%%% Local Variables: 
%%% mode: latex
%%% TeX-master: t
%%% End: 

% LocalWords:  maketitle KohDavGin pswads11 planetmath tntbase concl emph mkm05 lt92 ge
% LocalWords:  printbibliography newpart ednote hlt08 btc07 Lawvere thy-grph tn biform tp
% LocalWords:  seq ldots vdots noindent subtheory tikzpicture tdots thygraph cn funsat le
% LocalWords:  cdots ts1 ts2 tsdots tsn realm-mk realmref textbf realmref circ funsattac
% LocalWords:  compactenum highlevel wrapfigure vspace yscale textsf KohIan redge rcedge
% LocalWords:  ssmk12 defemph subseteq tpc12 RabKoh hookrightarrow xscale leadsto atp ci2
%  LocalWords:  KarMer ftg79 ttg59 slcirc circsl scriptscriptstyle invsl defeq mmtthy rm
%  LocalWords:  flatthy boxedquote varphi varphi bigraph mmtar medskip includeleft nmmtar
%  LocalWords:  pviewleft Kleisli mpd11 inparaenum overline lsl esl mapsto mapsto mapsto
%  LocalWords:  StaKoh tlcspx10 ppte12 Rabe omdoc orga CorTeX
