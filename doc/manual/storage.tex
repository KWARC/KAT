\section{Storage}

The annotation tool is storage agnostic per-se. As no back-end platform is provided
alongside the application, administrators are free to develop their own custom workflows
of storing an annotation. By default annotations are stored in the user's browser database
(or localStorage if the browser is older) and can be automatically transferred to a given
URI through a POST request. Annotations can also be loaded through GET requests from a
configurable URI or from the user's local storage.

Although the system is flexible in this regard, we recommend the storage of annotations in
a triple store as it provides a nicely continuity with the existing ontology based model.
The annotations can be easily represented as RDF structures and the application itself
stores it as so internally. Furthermore we believe that graph database are best suited for
storing and querying this kind of metadata.


%%% Local Variables: 
%%% mode: latex
%%% TeX-master: "katmanual"
%%% End: 
