\section{Developer Guide}
\KAT is written using FlancheJS: a simple library that provides javascript classical
inheritance. The classes are structured java-like, having:
\begin{itemize}
 \item constructors: provided by the method \textit{\textit{init}}.
 \item properties: the library automatically generates getters and setters.
 \item methods: publicly available functions.
 \item internals: properties and methods available only by underscore (\textbf{\textit{\_}}) prefixing.
\end{itemize}

The table below describes the \KAT classes (available under \textbf{src/js/}) and their behavior:

\begin{longtable}{p{6cm}|p{8cm}}%|
 \large{\textbf{Class Name}} & \large{\textbf{Class Description}}\\\hline
 \textbf{main} namespace: \\\hline
 \textit{kat.main.KATService} & The main entry point of the service. The \KAT Service requires a CSS3/XPATH selector to identify the container on which annotations can be made,
 and optionally a CorTeX service url and a document identifier for the annotated document.\\\hline
 \textbf{annotation} namespace: \\\hline
 \textit{kat.annotation.Annotation} & Describes an annotation that was collected from a user and can be saved and transported over
 network.\\\hline
 \textit{kat.annotation.AnnotationRegistry} & Describes an annotation registry that keeps track of all the annotations for the current document \\\hline
 \textit{kat.annotation.Concept} & Class to describe an annotation concept. Annotation concepts describe the annotation model (i.e. the fields contained
 by the annotation) and the behavior of the annotation (i.e. user interaction and display).\\\hline
 \textit{kat.annotation.ConceptRegistry} & A registry to keep track of all available concepts for this document.\\\hline
 \textit{kat.annotation.Ontology} & Class to describe an annotation ontology. Annotation ontologies describe the annotation concepts.\\\hline
 \textit{kat.annotation.OntologyRegistry} & A registry to keep track of all available ontologies for this document.\\\hline
 \textit{kat.annotation.AnnotationForm} & Class for handling the form displayed when an annotation is added.\\\hline
 \textbf{display} namespace: \\\hline
 \textit{kat.display.AnnotationEditForm} & Class for handling the form displayed when an annotation is edited.\\\hline
 \textit{kat.display.ControlPanel} & This class provides a tool for displaying and handling the \KAT Control Panel.\\\hline
 \textit{kat.display.AnnotationRenderer} & Class for handling the display of a singe annotation.\\\hline
 \textit{kat.display.ArrowConnector} &  Creates an svg arrow that can be used to connect two DOM elements, for example a reference field annotation to the referenced item.\\\hline
 \textit{kat.display.Display} & Creates and controls the annotation displays.\\\hline
 \large{\textbf{Class Name}} & \large{\textbf{Class Description}}\\\hline
 \textbf{input} namespace: \\\hline
 \textit{kat.input.Form} & The Form class decides which fields to be displayed in the current form. \\\hline
 \textit{kat.input.FormParser} & A form parser can be used to parse the fields and documentation from a given concept object.\\\hline
 \textit{kat.input.FieldParserRegistry} & The Field Parser Registry contains all the field parsers that are available to parse for a concept.\\\hline
 \textbf{input.fieldparser} namespace: \\\hline
 \textit{kat.input.fieldparser.FieldParser} & A field parser parses an annotation:field into an html string. This trait serves only as an interface that the extending classes follow. \\\hline
 \textit{kat.input.fieldparser.Checkboxes} & Parses a field of type checkboxes outputting html. \\\hline
 \textit{kat.input.fieldparser.Reference} & Parses a field of type reference outputting html. \\\hline
 \textit{kat.input.fieldparser.Select} & Parses a field of type select outputting html. \\\hline
 \textit{kat.input.fieldparser.TextArea} & Parses a field of type text area outputting html. \\\hline
 \textit{kat.input.fieldparser.TextField} & Parses a field of type text outputting html. \\\hline
 \textbf{input.fieldparser} namespace: \\\hline
 \textit{kat.input.view.Form} & Class that renders an annotation form containing the fields described in the concept. \\\hline
 \textit{kat.input.view.ConceptSelector} & Class to describe an input element in the form container that is used to select a concept to be used in the annotation form.\\\hline
 \textit{kat.input.view.FormContainer} & Describes a class that acts as a container for an annotation form and a concept selector.\\\hline
 \textbf{preprocessor} namespace: \\\hline
 \textit{kat.preprocessor.TextPreprocessor} & Used to add counters around text but this functionality has been deprecated. Now it only adds selection listeners in the text, for adding annotations.\\\hline
 \textbf{remote} namespace: \\\hline
 \textit{kat.remote.CoreTexAnnotationInserter} & Sends the annotations being created on this document to the CoreTeX system.\\\hline
 \textit{kat.remote.CoreTeXAnnotationReceiver} & Retrieves the document and the annotations from the CoreTeX service and populates the internal registry. \\\hline
\end{longtable}


%%% Local Variables: 
%%% mode: latex
%%% TeX-master: "kat"
%%% End: 
