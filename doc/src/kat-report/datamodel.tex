\section{\KAT System \& Information Architecture}\label{sec:sysarch}

\begin{figure}[ht]
\begin{tikzpicture}[yscale=1.5]
\tikzstyle{doc}=[draw,align=center,color=black,shape=document]
\tikzstyle{database}=[cylinder,shape border rotate=90,aspect=0.25,draw, 
     cylinder uses custom fill,cylinder body fill=yellow!30,cylinder end fill=yellow!30]
  \pgfdeclareimage[width=1cm]{user}{user};
  \node (user) at (-1,2) {\pgfuseimage{user}};
  \node[doc] (onto) at (6,2) {\begin{tabular}{c}Annotation\\Ontology\end{tabular}};
  \node[database] (docstore) at (0,0) {\begin{tabular}{c}Document
      \\Store\\\hline tHTML5 \end{tabular}};
  \node[database] (semblack) at (5,0) {\begin{tabular}{c}Semantic
      \\Blackboard\\\hline RDF\end{tabular}};
  \node[draw,rounded corners] (kat) at (2.5,2) {\begin{tabular}{c}KAT\\Annotator \end{tabular}};
  \draw[->] (docstore) -- node[left]{import} (kat);
  \draw[<-] (kat) to[out=-65,in=155] node[left]{import} (semblack);
  \draw[->] (kat) to[out=-25,in=105] node[right,near end]{export} (semblack);
  \draw[->,dotted] (semblack) -- node[above]{references} (docstore);
  \draw[<->] (user) -- node[above] {interact} (kat);
  \draw[->] (onto) -- node[above]{read} (kat);
\end{tikzpicture}
\caption{The \KAT System Architecture}\label{fig:kat-arch}
\end{figure}

\section{\KAT Annotation Data Model}
The annotations that the users will provide must be well structured in order for our third
requirement to be accomplished. This assures us that the work of annotators can best be
put to use by automatic processors or computer interpretors.

A good example of a use case is a self-learning software that translates text fragments
from one language to another.  For it to function properly it needs to know the
grammatical structure of the languages and here is where annotation tools can help.  Users
can annotate predefined sentences or paragraphs and identify specific parts of speech
(e.g. a noun) that the translation tool software developers can then use as a training
material for their machine-learning product.

Our system is flexible enough for administrators to
define how annotations should be structured and what connections can be made between
different annotations.  Furthermore this flexibility is also be applicable in the
visual interfaces with which users interact. In a typical system you will have users
coming from different backgrounds with various skills and interests. We believe that this
variety should not be suppressed but rather encouraged by allowing for multiple annotation
interfaces to be presented to the user so that he can choose the best suited one for his
contributions. This feature helps projects where crowdsourcing is detrimental by making
the users comfortable with the UI and allowing them to become proficient without a steep
learning curve.

\subsection{Concept Architecture}
At the center of our system is the \textbf{annotation concept}, several of which being
described by a user-supplied ontology. The format in which the ontology should be provided
is the standard OWL model.  Each annotation concept should correspond to an OWL instance
that is a member of the annotation class to which the following elements should be added:
\begin{itemize}
\item fields - this section describes how a user can input the necessary information for
  the annotation to be valid according to its concept. The section should contain for each
  field describing the annotation concept an entry that describes how a user can populate
  this information in a form.
  
  Each child of the entry should be of the following form:
  \begin{itemize}
  \item filed - the field wrapper, all further options are children of this element. It
    has \textit{name} and \textit{type} as attributes. The \textit{type} can be one of:
    text, select, reference, radios or text area.
  \item documentation - further information about the field, to be displayed to the user.
  \item value - a default value for the field.
  \item validation - a regular expression that the user input must match.
  \item number - a field having 2 attributes: \textit{atleast} and \textit{atmost},
    indicating how many inputs of this type can the user make in the same annotation.
  \item options [only available for field type = "select"] - the options from which the
    user should choose from. Each new option is an individual element having children of
    type \textit{documentation} and \textit{value}.
  \item referencedType [only available for field type = "reference"] - indicates the
    concept name that should be referenced by this input.
  \end{itemize}
\item display - this section describes how an annotation is displayed. It consist of the
  following fields:
  \begin{itemize}
  \item template - an HTML string containing the desired display format of the annotation
    fields. Each annotation field that is mentioned between curly braces will be replaced
    by the actual value of the field.
  \item style - a list of CSS valid rules to be applied to the annotation display.
  \end{itemize}
\end{itemize}

In Appendix A2 you can find an example of an annotation ontology. It defines 2 concepts
which allow the user to annotate text as Symbols or definitions of Symbols. As you can see
in the example, the model gives us the necessary flexibility at the administrator level
while at the same time providing for a clear customizable method of displaying the
annotation form and representation for the user.

%%% Local Variables: 
%%% mode: latex
%%% TeX-master: "kat"
%%% End: 
