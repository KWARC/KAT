\section{Introduction}\label{sec:intro}
Text annotation is the practice of adding a note to a text, which may include highlights
or underlining, comments, footnotes or links.  In most of the cases,annotations can be
thought of as text-metadata because they are usually added post hoc and provide
information about the text without fundamentally altering it.

A web annotation is an online annotation associated with a web resource. The annotation of
web-based documents by user communities is a widely used method of augmenting and adding
value to these resources and there are numerous use cases where the process can
disambiguate contexts or improve the overall readability.

In this report we present the \KAT annotation tool for (X)HTML documents, built with the
following objectives:
\begin{enumerate}
\item \emph{Usability} -- The tool should be convenient and easy to use by regular
  internet users allowing them varying degrees of complexity in the user interface.
\item \emph{Ease of integration} -- The tool should be easy to integrate into existing
  projects, and should have minimal to none dependencies on the server side.
\item \emph{Semantic Richness} -- The annotations provided by our users should contain the
  necessary level of information for NLP tools to use.
\end{enumerate}

\KAT is open source, it is licensed under the Gnu Public License~\cite{GPL:on} and hosted
on GitHub~\cite{KAT:github:on}. 

Development was started in Spring 2011 by Alex Mircea Dumitru and Vlad Merticariu as a
Computational Semantics Project at Jacobs University under the supervision of Deyan Ginev
and Michael Kohlhase. Stefan Mirea contributed the firefox plugin and the reviewing
functionality, and Tom Wiesing added JOBAD integration and general interface polish.

In the next section we review the state of the art in linguistic/semantic annotation
systems, \ednote{MK: continue} Section~\ref{sec:concl} concludes this report.

\paragraph{Acknowledgements}
The development of the \KAT was partially supported by the Leibniz association under grant
SAW-2012-FIZ\_KA-2.

%%% Local Variables: 
%%% mode: latex
%%% TeX-master: "kat"
%%% End: 

%  LocalWords:  emph Merticariu Mirea Wiesing ednote concl kat
