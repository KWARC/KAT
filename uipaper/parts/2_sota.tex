\section{State Of The Art}
	Currently there are several annotations tools available to use for online text annotations including Hypothes.is, brat, Yawas and Annotatie \cite{stateOfTheArt}. Most state of the art anotation tools create one of the following types of annotations: \ednote{decide whether to put his in intro or here}
	  \begin{enumerate}
	    \item \textit{Dynamic Annotations} - These create annotations that are anchored to the text of the document.
	    \item \textit{Static Annotations} - These create annotations that are anchored to a particular position in the page of the document.
	  \end{enumerate}

	\subsection{Hypothes.is}
    	Hypothes.is \cite{hypothesis} is a tool developed with the aim of ``adding a new layer to the web''. This tool is an online, dynamic annotation tool that can highlight and annotate pdfs and web pages. It provides additional features such as the ability to make an annotation public or private, and being able to post replies to annotations. A user needs to create an account before they can create an annotation, enabling Hypothes.is to show the user his/her private annotations each time they access the webpage. The tool can be either downloaded or run as a Javascript plugin.    

    \subsection{brat}
	    The brat rapid annotation tool \cite{brat} is a web based dynamic text annotation tool. It is designed to create annotations that have a fixed form that can be automatically processed and interpreted. brat can handle two basic types of annotations:
	    \begin{enumerate}
	      \item Text Span Annotations: Creates a simple annotation on a stretch of highlighted text.
	      \item Relation Annotations: Creates a connection between two text-span annotations.
	    \end{enumerate}

	    brat provides several functionalities that make it easy to use. The main ones are:
	    \begin{enumerate}
	      \item Advanced annotation searching tool.
	      \item An annotation export interface that can convert the internal storage format to PDF or HTML.
	      \item Unique address to access each annotation.
	    \end{enumerate}

  	\subsection{Weaknesses Identified in SotA}
    	\paragraph{Inability to handle structured documents}
    		When creating a new annotation, we need to mark the text that the annotation refers to. The state of the art annotation tools are equipped to handle only plain-text documents. While this makes storing annotations very simple as we just store the positions of the start and end characters in a string, it also makes it unsuitable for STEM annotation as the structure carries important syntactic and semantic information
    	\paragraph{Visually disruptive annotations}
    		One key weakness with state of the art annotation tools is the fact that certain annotations break the flow of the underlying text. This is particularly true of relation annotations in brat. If there are several relation annotations connecting two annotations that are separated by a multiple lines, the layout of the document is no longer user friendly.
    	\paragraph{Insufficient help for new users}
    		In order to improve user experience and system usability there needs to be more help provided to help new users understand how to use the new system. While brat provided a tutorial, none of the other 3 systems had any sort of help or guide. Yet even in brat, finding the tutorial was not intuitive.